\documentclass[a4paper,
fontsize=11pt,
%headings=small,
oneside,
numbers=noperiodatend,
parskip=half-,
bibliography=totoc,
final
]{scrartcl}

\usepackage{synttree}
\usepackage{graphicx}
\setkeys{Gin}{width=.4\textwidth} %default pics size

\graphicspath{{./plots/}}
\usepackage[ngerman]{babel}
\usepackage[T1]{fontenc}
%\usepackage{amsmath}
\usepackage[utf8x]{inputenc}
\usepackage [hyphens]{url}
\usepackage{booktabs} 
\usepackage[left=2.4cm,right=2.4cm,top=2.3cm,bottom=2cm,includeheadfoot]{geometry}
\usepackage{eurosym}
\usepackage{multirow}
\usepackage[ngerman]{varioref}
\setcapindent{1em}
\renewcommand{\labelitemi}{--}
\usepackage{paralist}
\usepackage{pdfpages}
\usepackage{lscape}
\usepackage{float}
\usepackage{acronym}
\usepackage{eurosym}
\usepackage[babel]{csquotes}
\usepackage{longtable,lscape}
\usepackage{mathpazo}
\usepackage[normalem]{ulem} %emphasize weiterhin kursiv
\usepackage[flushmargin,ragged]{footmisc} % left align footnote
\usepackage{ccicons} 

%%%% fancy LIBREAS URL color 
\usepackage{xcolor}
\definecolor{libreas}{RGB}{112,0,0}

\usepackage{listings}

\urlstyle{same}  % don't use monospace font for urls

\usepackage[fleqn]{amsmath}

%adjust fontsize for part

\usepackage{sectsty}
\partfont{\large}

%Das BibTeX-Zeichen mit \BibTeX setzen:
\def\symbol#1{\char #1\relax}
\def\bsl{{\tt\symbol{'134}}}
\def\BibTeX{{\rm B\kern-.05em{\sc i\kern-.025em b}\kern-.08em
    T\kern-.1667em\lower.7ex\hbox{E}\kern-.125emX}}

\usepackage{fancyhdr}
\fancyhf{}
\pagestyle{fancyplain}
\fancyhead[R]{\thepage}

% make sure bookmarks are created eventough sections are not numbered!
% uncommend if sections are numbered (bookmarks created by default)
\makeatletter
\renewcommand\@seccntformat[1]{}
\makeatother


\usepackage{hyperxmp}
\usepackage[colorlinks, linkcolor=black,citecolor=black, urlcolor=libreas,
breaklinks= true,bookmarks=true,bookmarksopen=true]{hyperref}
%URLs hart brechen
\makeatletter 
\g@addto@macro\UrlBreaks{ 
  \do\a\do\b\do\c\do\d\do\e\do\f\do\g\do\h\do\i\do\j 
  \do\k\do\l\do\m\do\n\do\o\do\p\do\q\do\r\do\s\do\t 
  \do\u\do\v\do\w\do\x\do\y\do\z\do\&\do\1\do\2\do\3 
  \do\4\do\5\do\6\do\7\do\8\do\9\do\0} 
% \def\do@url@hyp{\do\-} 
\makeatother 

%meta
%meta

\fancyhead[L]{Redaktion LIBREAS \\ %author
LIBREAS. Library Ideas, 33 (2018). % journal, issue, volume.
\href{http://nbn-resolving.de/}
{}} % urn 
% recommended use
%\href{http://nbn-resolving.de/}{\color{black}{urn:nbn:de...}}
\fancyhead[R]{\thepage} %page number
\fancyfoot[L] {\ccLogo \ccAttribution\ \href{https://creativecommons.org/licenses/by/3.0/}{\color{black}Creative Commons BY 3.0}}  %licence
\fancyfoot[R] {ISSN: 1860-7950}

\title{\LARGE{Das liest die LIBREAS, Nr. \#2 (Frühjahr 2018)}} % title
\author{Redaktion LIBREAS} % author

\setcounter{page}{1}

\hypersetup{%
      pdftitle={Das liest die LIBREAS, Nr. \#2 (Frühjahr 2018)},
      pdfauthor={Redaktion LIBREAS},
      pdfcopyright={CC BY 3.0 Unported},
      pdfsubject={LIBREAS. Library Ideas, 33 (2018).},
      pdfkeywords={Open Access},
      pdflicenseurl={https://creativecommons.org/licenses/by/3.0/},
      pdfcontacturl={http://libreas.eu},
      baseurl={http://libreas.eu},
      pdflang={de},
      pdfmetalang={de}
     }



\date{}
\begin{document}

\maketitle
\thispagestyle{fancyplain} 

%abstracts

%body
Beiträge von Linda Freyberg (lf), Ben Kaden (bk), Karsten Schuldt (ks),
Michaela Voigt (mv), Viola Voß (vv)

\hypertarget{zur-kolumne}{%
\section{Zur Kolumne}\label{zur-kolumne}}

Das Ziel dieser Kolumne ist, eine Übersicht über die in der letzten Zeit
erschienene bibliothekarische, informations- und
bibliothekswissenschaftliche sowie für diesen Bereich interessante
Literatur zu geben. Enthalten sind Beiträge, die der LIBREAS-Redaktion
oder anderen Beitragenden als relevant erschienen.

Themenvielfalt sowie ein Nebeneinander von wissenschaftlichen und
nicht-wissenschaftlichen Ansätzen wird angestrebt. Auch in der Form
sollen traditionelle Publikationen ebenso erwähnt werden wie
Blogbeiträge oder Videos beziehungsweise TV-Beiträge.

Gern gesehen sind Hinweise auf erschienene Literatur oder Beiträge in
anderen Formaten. Die Redaktion freut sich über entsprechende Hinweise
(siehe \url{http://libreas.eu/about/}, Mailkontakt für diese Kolumne ist
\href{mailto:zeitschriftenschau@libreas.eu}{\nolinkurl{zeitschriftenschau@libreas.eu}}).
Die Koordination der Kolumne liegt bei Karsten Schuldt. Verantwortlich
für die Inhalte sind die jeweiligen Beitragenden. Die Kolumne
unterstützt den Vereinszweck des LIBREAS-Vereins zur Förderung der
bibliotheks- und informationswissenschaftlichen Kommunikation.

LIBREAS liest gern und viel Open-Access-Veröffentlichungen. Wenn sich
Beiträge doch einmal hinter eine Bezahlschranke verbergen, werden diese
durch \enquote{{[}Paywall{]}} gekennzeichnet. Zwar macht das Plugin
Unpaywall (\url{http://unpaywall.org/}) das Finden von legalen
Open-Access-Versionen sehr viel einfacher. Als Service an der
Leserschaft verlinken wir OA-Versionen, die wir vorab finden konnten,
jedoch nach Möglichkeit auch direkt. Für alle Beiträge, die nicht frei
zugänglich sind, empfiehlt die Redaktion Werkzeuge wie den Open Access
Button (\url{https://openaccessbutton.org/}) zu nutzen oder auf Twitter
mit \#icanhazpdf (\url{https://twitter.com/hashtag/icanhazpdf?src=hash})
um Hilfe bei der legalen Dokumentenbeschaffung zu bitten.

\hypertarget{artikel-und-zeitschriftenausgaben}{%
\section{Artikel und
Zeitschriftenausgaben}\label{artikel-und-zeitschriftenausgaben}}

Ein Forschungsseminar \enquote{Bibliothek der Zukunft} im Studiengang
Bibliotheks- und Informationswissenschaft an der HTKW Leipzig hat sich
mit dem Thema \enquote{Embedded Libarianship} (EL) beschäftigt (Figge,
Friedrich / Darby, Kisten / Hardt, Jens / Liebig, Theresa / Remeli,
Eva-Maria / Wilde, Viviane (2017): „Neue Aufgaben, neue Arbeitsfelder,
neue Strukturen. Zur Zukunft der Wissenschaftlichen Bibliotheken im
internationalen Forschungswettbewerb am Beispiel des Embedded
Librarian\enquote{. In: \emph{BuB - Forum für Bibliothek und
Information} 69 (10) :558-561 {[}Paywall{]}). Der Artikel gibt einen
guten Überblick darüber, welche Aufgaben für}eingebettete"
Bibliothekar:innen denkbar sind, an welchen Stellen im
Wissenschaftskreislauf sie eingebunden werden können und wie die
Organisationsstruktur einer wissenschaftlichen Bibliothek aussehen
könnte, die EL einführen will. Aus Sicht von Praktiker'innen bietet
dieses theoretische \enquote{Gedankenexperiment} (S. 559) allerdings
diverse Fragen zur Umsetzbarkeit -- und damit Stoff für weitere
Veröffentlichungen. {[}Längere Fassung der Besprechung:
\url{https://libreas.wordpress.com/2018/02/09/embedded-librarianship/}{]}
(vv)

Ein Nature-Editorial berichtet von zwei Fällen, in denen Haustiere als
Koautoren wissenschaftlicher Aufsätze angegeben wurden. Neben der
berühmten Katze Chester, die unter F. D. C. Willard als Zweitautor eines
Papers in Physical Review Letters galt
(\url{https://doi.org/10.1103/PhysRevLett.35.1442}), wird von einem
Hamster berichtet, der als Koautor des Papers \enquote{Detection of
earth rotation with a diamagnetically levitating gyroscope} in Physica
B: Condensed Matter genannt wird (A.K.Geim, H.A.M.S.ter Tisha;
\url{https://doi.org/10.1016/S0921-4526(00)00753-5}) (Editorial: From
proposals to snarks: the messages that scientists sneak into their
papers. In: \emph{Nature} 554, 276 (2018) / 14.02.2018
\url{https://doi.org/10.1038/d41586-018-01876-8}) (bk)

\enquote{The lies we tell ourselves}. Fobazi Ettarh befragt -- sehr auf
die USA bezogen, daher nur mit Einschränkungen auf den DACH-Raum
übertragbar -- die bibliothekarische Profession darauf, ob diese sich
nicht immer wieder selber eine falsche Erzählung von den Aufgaben und
\enquote{der Berufung}, Bibliothekarin oder Bibliothekar zu sein,
präsentiert. Die Bibliothek als sozialer Ort, als sicherer Ort, als
demokratischer Ort, als Wissensort und die bibliothekarische Profession
als eine inhärent gute scheinen ihm einen religiösen Hintergrund zu
haben, der mit der eigentlichen, alltäglichen Arbeit wenig zu tun hat.
Diese Differenz zwischen Anspruch und Realität würde aber benutzt, um
die Löhne gering zu halten (weil es eine Berufung wäre) und sich wenig
um das Wohl der Bibliothekspersonals zu kümmern. Ergebnis seien
Burn-Out, Reproduktion von sozialen Strukturen und das ständige
Anwachsen von Arbeit in den Bibliotheken. Wenig überraschend löste der
provokante Text eine einigermassen rege Diskussion aus. (Ettarh, Fobazi:
Vocational Awe and Librarianship. The Lies We Tell Ourselves. In:
\emph{In The Library With The Lead Pipe}, 10.01.2018,
\url{http://www.inthelibrarywiththeleadpipe.org/2018/vocational-awe/})
(ks)

Unzufrieden mit der Erkenntnis, dass Katalogisierung auch immer die
Macht beinhaltet, Wissen ein- oder auszuschliessen und vom Wunsch
getrieben, Wissen so zu präsentieren, wie Nutzerinnen und Nutzer dieses
Wissen selber suchen, stellt Jessica L. Colbert Überlegungen zu
\enquote{Patron-Driven Subject Access} an. Sie postuliert, dass aktuelle
Kataloge eigentlich dafür ausgestattet seien, beispielsweise per
Tagging, den Nutzenden einen grösseren Einfluss darauf zu geben, wie
diese Kataloge organisiert sind. Dazu müssten Bibliotheken aber aktiv
werden und die Nutzenden auch beteiligen. (Jessica L. Colbert:
Patro-Driven Subject Access: How Librarians Can Mitigate That
\enquote{Power To Name}. In: \emph{In The Library With The Lead Pipe},
15.10.2017,
\url{http://www.inthelibrarywiththeleadpipe.org/2017/patron-driven-subject-access-how-librarians-can-mitigate-that-power-to-name/})
(ks)

Eine teilweise radikale Kritik am \enquote{Innovations-Fetisch} in
Bibliotheken -- also dem Phänomen, Innovation als Wert an sich zu
verstehen, der wichtiger gewertet wird als gute bibliothekarische Arbeit
-- führt Julia Glassman durch. Dieser Fetisch sei schlecht für die
Bibliotheken, das Personal und auch die Nutzerinnen und Nutzer; ein
Ausdruck eines gewissen kapitalistischen Denkens, das am eigentlichen
Sinn von Bibliotheken vorbeizielen würde. Es sei notwendig, sich gegen
diesen Fetisch zu stellen. (Julia Glassman: The Innovation Fetish and
Slow Librarianship. What Librarians Can Learn From the Juicero. In:
\emph{In The Library With The Lead Pipe}, 18.10.2017,
\url{http://www.inthelibrarywiththeleadpipe.org/2017/the-innovation-fetish-and-slow-librarianship-what-librarians-can-learn-from-the-juicero/})
(ks)

Schwerpunkt der Dezember-2017-Ausgabe des Journal of the European
Association for Health Information and Libraries sind konsortiale
Lizenzierungslösungen für Datenbanken und elektronische Medien,
teilweise im nationalen Rahmen. Selbstverständlich bezogen auf
Medizinbibliotheken bieten sie trotzdem eine Übersicht zu den
verschiedenen Ansätzen. (\emph{Journal of EAHIL} 13 (2017) 4,
\url{http://eahil.eu/wp-content/uploads/2016/05/journal-4-2017-web.pdf})
(ks)

Auf ihrem Vortrag auf dem letzten Bibliothekstag (2017, Frankfurt am
Main) aufbauend berichtet Heidrun Wiesenmüller über die
Weiterentwicklungen von RDA. Der Artikel ist kritisch vor allem
gegenüber der Geschwindigkeit, mit der diese Änderungen vollzogen
werden. Gleichzeitig macht er klar, dass es kein Ende der ständigen
Umbauten in RDA geben wird. RDA wird fortgeschrieben werden, wie
Software, mit ständigen Versionierungen, einem Steuerungsboard und von
Zeit zu Zeit einem radikalen Umschreiben des gesamten Standards.
(Wiesenmüller, Heidrun: Baustelle RDA -- die Dynamik des Regelwerks als
Herausforderung. In: \emph{o\textbar{}bib} 4 (2017) 4,
\url{https://doi.org/10.5282/o-bib/2017H4S176-188}) (ks)

Wie schwierig das Ideal eines offenen Zugangs zu Forschungsdaten im
Sinne der Open Science praktisch einzulösen ist, wird an einem Editorial
in Nature deutlich. Dieses verweist ein weiteres Mal auf die
Unverzichtbarkeit von Forschungsdatenmanagementplänen, die, so die
Nature-Redaktion, vor allem durch Forschungsförderer nachdrücklich
kommuniziert werden müssen. Dabei gilt es, die jeweiligen Besonderheiten
der unterschiedlichen Forschungsfelder und -communities zu beachten.
Disziplin-übergreifende Standards werden kaum durchsetzbar sein.
Insgesamt scheinen Forschende erstaunlich wenig interessiert und
qualifiziert für die konkreten Herausforderungen des
Forschungsdatenmanagements. Dies dürfte auch jede Open-Science-Bewegung
massiv bremsen, denn erst gut aufbereitete, annotierte und mit
entsprechenden Metadaten versehene Forschungsdaten eignen sich auch für
eine mögliche Nachnutzung. (o.A.: Making Plans. They sound dull, but
data-management plans are essential, and funders must explain why. In:
\emph{Nature} 555, 286 15.03.2018, 286
\url{https://doi.org/10.1038/d41586-018-03065-z}) (bk)

\hypertarget{die-sicht-der-nutzerinnen-und-nutzer}{%
\subsection{Die Sicht der Nutzerinnen und
Nutzer}\label{die-sicht-der-nutzerinnen-und-nutzer}}

Eine amüsante, aber auch bedenkenswerte (und in einzelnen Bibliotheken
wiederholbare) Studie zur Differenz zwischen dem, was Studierende an
einer Universität (University of Mississippi) von ihrer Bibliothek
wollen und was Bibliothekarinnen und Bibliothekare denken, dass sie
wollen, legten Young und Kelly vor. Sie hatten in einer vorhergehenden
Studie mittels einer Card-Sorting-Methode die Vorstellungen der
Studierenden erhoben, jetzt erhoben sie mit der gleichen Methode die des
Bibliothekspersonals. Die Unterschiede waren teilweise immens. Gerade
der Wunsch nach ausborgbaren Taschenrechnern wurde massiv unterschätzt,
dafür der nach spezifischen Druckmöglichkeiten überschätzt. Sicherlich
sind die Ergebnisse spezifisch für die untersuchte Bibliothek, auch über
die Methode wird sich streiten lassen. Die Differenzen zwischen den
beiden Gruppen werden sich aber in anderen Bibliotheken und in anderen
Sprachräumen finden. (Young, Brian W. ; Kelley, Savannah L.: How Well Do
We Know Our Students? A Comparison of Students' Priorities for Services
and Librarians' Perceptions of Those Priorities. In: \emph{The Journal
of Academic Librarianship} in Print,
\url{https://doi.org/10.1016/j.acalib.2018.02.010} {[}Paywall{]}) (ks)

Nachdem in den letzten Jahren Hochschulbibliotheken zu flexiblen
Lernlandschaften umgebaut wurden, scheint jetzt folgerichtig eine
Beschäftigung von Bibliotheken zu sein, zu untersuchen, ob diese von den
Studierenden -- für die diese Umbauten vorgenommen wurden -- genutzt
werden. Ein Beispiel dafür ist die Studie an der Loyola Marymount
University, Californien. (Gardner Archambault, Susan \& Justice,
Alexander: Student Use of the Information Commons: An Exploration
through Mixed Methods. In: \emph{Evidence Based Library and Information
Practice} 12 (2017) 4, \url{https://doi.org/10.18438/B8VD45}) Sie stellt
einerseits die Methoden vor, welche sich in letzter Zeit für solche
Untersuchungen etabliert zu haben scheint (Beobachtungen, Befragungen,
Mitmach-Fragen), andererseits Ergebnisse, die sich ebenso immer wieder
zeigen, nämlich dass die Studierenden die flexiblen Angebote und die
Möglichkeiten zur Gruppenarbeit kaum nutzen, sondern vielmehr das
solitäre Arbeiten und Lernen bevorzugen und gleichzeitig sich so sicher
und wohl führen, dass sie die genutzten Plätze temporär zu privaten
Räumen umbauen. (ks)

Als Beitrag zur Methodenentwicklung in der Bibliotheksforschung und den
Versuchen, die Sicht der Nutzerinnen und Nutzer in die Planung von
bibliothekarischen Angeboten und Bibliotheksräumen einzubeziehen,
verstehen Shailoo Bedi und Jenaya Webb ihre Übersicht über die
Verwendung von Fotografien in quasi-ethnologischen Interviews, die von
Bibliotheken durchgeführt werden. Die Methode würde zu einem besseren
Verständnis davon führen, was Nutzerinnen und Nutzer im Raum wahrnehmen.
(Bedi, Shailoo ; Webb, Jenaya: Participant-driven photo-elicitation in
library settings: A methodological discussion. In: \emph{Library and
Information Research} 41 (2017) 125,
\url{http://www.lirgjournal.org.uk/lir/ojs/index.php/lir/article/view/752})
(ks)

\hypertarget{open-access}{%
\subsection{Open Access}\label{open-access}}

Eine kritische Auseinandersetzung mit dem Projekt DEAL lieferten Alex
Holcombe und Björn Brembs im Dezember 2017: Ihrer Meinung nach wäre der
deutschen Wissenschaftslandschaft mit einem Scheitern der
DEAL-Verhandlungen mit Elsevier besser gedient. Holcombe und Brembs
argumentieren, dass durch einen nationalen Vertrag AutorInnen (erneut)
nicht mit den Kosten für das OA-Publizieren konfrontiert wären -- analog
zu den Verhältnissen in der Subskriptionswelt. Folglich wäre der Journal
Impact Factor weiter das ausschlaggebende Auswahlkriterium -- denn die
Forschungsetats blieben von den OA-Gebühren augenscheinlich unberührt.
Infolge dessen würde Verlage die Preise weiterhin basierend auf dem
Prestige der Journale (anstatt auf den tatsächlich anfallenden Kosten)
bemessen, so dass ein DEAL de facto nicht die angestrebte Disruption des
Publikationsmarktes bedeuten würde. Beide Autoren wünschen sich daher
einen drastischeren Umschwung -- weg von traditionellen Verlagen und
Publikationsweisen, hin zu innovativen Publikationsformen, die durch die
Wissenschaft selbst organisiert werden. (Alex Holcombe, Björn Brembs:
Open access in Germany: the best DEAL is no deal. In: \emph{Times Higher
Education's Blog}, 27.12.2017
\url{https://www.timeshighereducation.com/blog/open-access-germany-best-deal-no-deal})
(mv)

Einen kompakten Einstieg in das Thema Open-Access-Transformation liefert
Frank Scholze in einem Beitrag in der Festschrift für den ehemals
leitenden Direktor der ULB Darmstadt Hans-Georg Nolte-Fischer, welche
kürzlich bei Harrossowitz erschien. Auf übersichtlichen sechs Seiten
fasst Scholze verschiedene politische Positionen zum Thema zusammen und
stellt Transformationsansätze für die Bereiche OA Gold (am Beispiel KIT
Karlsruhe: integriertes Budget für Subskription und OA-Gebühren) und OA
Grün vor (am Beispiel von DeepGreen: das Projekt leiste wichtige Arbeit
in den Bereichen Standardisierung sowie Implementierung des
Datenaustauschs zwischen Verlagen und Repositorien). Sein Fazit: Die
(vollständige) Transformation des wissenschaftlichen Publikationsmarktes
ist unvermeidlich. Auf dem Weg dahin sind Nachhaltigkeit und
Angemessenheit von Kosten wichtige Aspekte. (Scholze, Frank:
Open-Access-Transformation. In: \emph{Vom Sinn der Bibliotheken.
Festschrift für Hans-Georg Nolte-Fischer}. Wiesbaden: Harrassowitz,
2017. S. 173--178. {[}Paywall{]} {[}OA-Version:
\url{https://nbn-resolving.org/urn:nbn:de:swb:90-754646}{]}) (mv)

Welche Kosten birgt die Herstellung und Verbreitung eines Buches? Dieser
Frage hat sich Francis Pinter gewidmet: Sie bespricht in ihrem Artikel
die Ergebnisse einer Vergleichsstudie von verschiedenen
(wissenschaftlichen Buch-) Verlagen aus Österreich, Dänemark, Finnland,
Frankreich, Deutschland, Niederlande, Norwegen und dem Vereinigtem
Königreich. Dabei wurden Angebote von Universitätsverlagen ebenso
untersucht wie die von etablierten Verlagen. Auch nach der Lektüre
bleibt die Frage offen, was der durchschnittlichen Preis für ein
(Open-Access-)Buch ist (zwischen 500 und 18.500 Euro). Aber Pinter
stellt schlüssig dar, warum es so schwer ist, Durchschnittspreise für
Bücher, insbesondere für Open-Access-Bücher, zu ermitteln --
unterschiedliche Geschäfts- und Kostenmodelle der Verlage sind ebenso
Gründe wie die Ausstattung und der anvisierte Markt. (Pinter, Francis:
Why Book Processing Charges (BPCs) Vary So Much. In: \emph{The Journal
of Electronic Publishing}, 21 (2018) 1,
\url{https://doi.org/10.3998/3336451.0021.101}) (mv)

Die letzten 20 Jahre haben gezeigt: Es reicht nicht, Repositorien als
technische Infrastruktur bereitzustellen -- für eine erfolgreiche
Umsetzung des grünen Weges von Open Access und die breite Nutzung von
Repositorien braucht es Services von Bibliotheken. Jonathan Bull und
Teresa Schultz stellen in ihrem Artikel einen semi-automatisierten
Workflow vor, den sie für die Valparaiso University entwickelt haben, um
die Contentakquise des institutionellen Repositoriums (technisches
Basis: Digital Commons von bepress\footnote{Die Firma bepress wurde 2017
  durch Elsevier aufgekauft. Die Bibliothek der University of
  Pennsylvania hat daraufhin angekündigt, die Partnerschaft mit bepress
  zu beenden und berichtet über den Umstieg zu einer neuen
  Repositorienlösung im eigens gestarteten Blog und Twitter-Account:
  \url{https://beprexit.wordpress.com/} und
  \url{https://twitter.com/beprexit/}.}) zu optimieren. Wer hofft, einen
fertigen Workflow vorzufinden, wird leider enttäuscht. Vielmehr
beschreibt der Artikel detailliert die getesteten Ansätze, wobei die
Akquise von Metadaten im Vordergrund steht. Die Akquise von Volltexten,
die ja eigentlich zentral für ein Open-Access-Repositorium sind, wird
als offenes Problem präsentiert. Wer jedoch als Repository-Betreiber
ganz am Anfang bei den Überlegungen steht, Prozesse zu optimieren und
automatisieren, kann sich von der umfangreichen Diskussion zu Problemen,
Fehlern und möglichen Ansätzen inspirieren lassen. (Bull, Jonathan \&
Schultz, Teresa A.: Harvesting the Academic Landscape: Streamlining the
Ingestion of Professional Scholarship Metadata into the Institutional
Repository. In: \emph{Journal of Librarianship and Scholarly
Communication}. 6 (2018) 1,
\url{https://doi.org/10.7710/2162-3309.2201}) (mv)

Eine interessante Perspektive auf die Langzeitarchivierung für digitale
Kunst, also auf Software basierenden Werken, schildern Deena Endel und
Glenn Wharton anhand von Erfahrungen im Museum of Modern Art (MoMA), New
York. Entscheidend ist die Dokumentation des jeweiligen Quellcodes,
woraus die Notwendigkeit für die Entwicklung passender
Software-Dokumentationen entsteht. Die Orientierung an den Praxen der
allgemeinen Software-Entwicklung hilft ein Stück weit, jedoch erfordern
die besonderen Bedingungen und Ansprüche von Software-Kunst Anpassungen.
So liegt ein Schwerpunkt auf der umfassenden und eindeutigen
Dokumentation der ästhetischen Komponenten des Werkes. Das Ziel der
Reproduzierbarkeit der ästhetischen Erfahrungen und der von der
Künstlerin/dem Künstler beabsichtigten Wirkung muss im Mittelpunkt der
Dokumentation stehen. Auf der anderen Seite ist die
Quellcode-Dokumentation auch aus Sicht einer digitalen Kunstgeschichte
beziehungsweise besser Geschichte der Digitalkunst eine höchst
aufschlussreiche Forschungsgrundlage. Sie ermöglicht es, sich den
Intentionen der Werkschaffenden zu nähern sowie ihre Intentionen und
Kompetenzen nachzuvollziehen. Deena Engel und Glenn Wharton schreiben
von einem \enquote{rich trove of hidden information {[}...{]} embedded
in source code}, der für eine kunstwissenschaftliche Auseinandersetzung
mit den jeweiligen Arbeiten ebenso relevant sein dürfte wie für die
Restauratoren. (Deena Engel und Glenn Wharton: Source Code Analysis as
Technical Art History. In: \emph{Journal of the American Institute for
Conservation}. 2 (2015), S. 91--101
\url{https://doi.org/10.1179/1945233015Y.0000000004} {[}Paywall{]}
{[}OA-Version:
http://authenticationinart.org/pdf/literature/1945233015Y.pdf{]}) (bk)

\hypertarget{uxf6ffentliche-bibliotheken-lesepraxen-und-lesefuxf6rderung}{%
\subsection{Öffentliche Bibliotheken, Lesepraxen und
Leseförderung}\label{uxf6ffentliche-bibliotheken-lesepraxen-und-lesefuxf6rderung}}

Mit einem etwas einfachen psychologischen Modell untersuchten Hanna
Schmidt und Kyra Hamilton die Überzeugungen, welche Personen in
Erziehungsverantwortung (caregivers) dazu bringen oder davon abhalten,
mit einem Kleinkind eine (australische) Bibliothek zu besuchen.
(Schmidt, Hanna ; Hamilton, Kyra: Caregivers' beliefs about library
visits: A theory-based study of formative research. In: \emph{Library
and Information Science Research} 39 (2017) 4, 267--275,
\url{https://doi.org/10.1016/j.lisr.2017.11.002} {[}Paywall{]})
Grundsätzlich überlagern sich positive und negative Vorstellungen; dies
jeweils in einem subjektiven Mix. Die Autorinnen schlagen vor, dass
Bibliotheken gezielt auf die negativen Vorstellungen eingehen, um diese
auszuräumen. (ks)

Einen umfassenden Überblick zu den Leseförderprogrammen von Öffentlichen
Bibliotheken in den Niederlanden, inklusive Auswertungen über deren
Wirkungen, liefern Adriaan Langendonk und Kees Broekhof. (Langendonk,
Adriaan ; Broekhof, Kees: The Art of Reading: The National Dutch Reading
Promotion Program. In: \emph{Public Library Quarterly} 36 (2017) 4,
293-317, \url{https://doi.org/10.1080/01616846.2017.1354351}
{[}Paywall{]}) Während der Text auf den ersten Blick umfassend ist und
auch mit einem positiven Fazit zum Tool \enquote{Library at School
Monitor} (eine kontinuierliche Befragung und Auswertung von Daten, die
mit den Programmen zu tun haben) endet, zeigt er auch die Grenzen all
der Forschungen und Vorstellungen dieser Leseförderprogramme an. Es wird
nicht mehr geklärt, was Lesen ist, was genau das Ziel von Leseförderung
ist (sollen die Schülerinnen und Schüler viel lesen, etwas Bestimmtes
lesen, bestimmte Kompetenzen erwerben?), es wird auch nicht nach den
Wirkungsweisen gefragt, sondern nur danach, ob es sich in messbaren
Zahlen niederschlägt (und die Arbeit, aus diesen Zahlen etwas zu machen,
auf die Personen vor Ort verschoben). Auf den zweiten Blick scheint der
Text klar zu markieren, dass bei grossen Leseförderprogrammen vor allen
das möglichst viele Lesen als Ziel gilt, ohne das breiter darüber
diskutiert würde, ob das sinnvoll ist. (ks)

Eine Studie zum Umgang mit der Frage nach dem geistigen Eigentum in den
Makerspaces in öffentlichen Bibliotheken ergab, dass Bibliothekar*innen
den Datenschutz (\emph{Patron Privacy}) der Nutzer*innen deutlicher zu
gewichten scheinen als die Überprüfung, ob diese das durch die
Bibliothek bereitgestellten Makerspace-Technologien auch konform zu den
rechtlichen Ansprüchen des \emph{Intellectual Property Law} anwenden.
Zudem besteht eine gewisse Scheu, konkrete Hinweise zu Rechtsfragen
dieser Art zu geben sowie diese in den Benutzungsordnungen für die
Makerspaces zu berücksichtigen. Dies erklärt sich teilweise aus dem
Wunsch, eventuelle Haftungsfragen zu vermeiden. Die Autor*innen
empfehlen dennoch nachdrücklich, dass die Vermittlung eines
entsprechenden Wissens zum Geistigen Eigentum Teil der Aus-
beziehungsweise Weiterbildung für Bibliotheksmitarbeiter*innen werden
sollte. Zudem sollten die Benutzungsregeln kontinuierlich parallel zur
Entwicklungen der in den Makerspaces angebotenen Technologien überprüft
und angepasst werden. (Bossaller, Jenny; Haggerty, Kenneth: We are not
police: Public librarians' attitudes about making and intellectual
property. In: \emph{Public Library Quarterly}. 37 (2018) 1, 36-52,
\url{https://doi.org/10.1080/01616846.2017.1422173} {[}Paywall{]}) (bk)

Eine Studie zu den literalen Praxen von Menschen, die (in den USA)
regelmäßig Parks nutzen -- zumeist, weil sie obdachlos sind -- lies die
Autorinnen und Autoren einer Studie zum Thema bemerken:
\enquote{{[}O{]}ur observational data and our researchers' reflections
show very clearly and surprising to us, that our participants are
engaging in numerous literacy activities daily. This finding
corroborates what Miller {[}...{]} has found among homeless people that,
by and large, they are not illiterate, indeed they are avid readers.
What homeless people lack are the resources and the social capital that
they need in order to rise above their circumstances.} Der Text geht
auch darauf ein, wie diese Obdachlosen die Universitätsbibliotheken im
Umfeld der jeweiligen Parks nutzen. Grundsätzlich sind sie interessierte
Leserinnen und Leser, die zum Teil nicht wissen, was ihnen an Angeboten
offen stehen würde und die Bibliotheken zum Teil als zu sehr überwacht
wahrnehmen. Die Studie zielt auf die USA, insoweit sind viele Ergebnisse
nicht direkt zu übertragen. Zu vermuten ist aber, dass die vielfältigen
literalen Praxen von Menschen ohne festen Wohnsitz sich auch in den
deutschsprachigen Staaten finden werden. (Tinker Sachs, Gertrude et al.:
Literacy scholars coming to know the people in the parks, their literacy
practices and support systems. In: \emph{Critical Inquiry in Language
Studies} 15 (2018) 1,
\url{https://doi.org/10.1080/15427587.2017.1351880} {[}Paywall{]}) (ks)

\hypertarget{critical-library-instruction}{%
\subsection{Critical Library
Instruction}\label{critical-library-instruction}}

Eine im englischsprachigen Raum verbreitete Diskussion, auch Praxis, die
gleichzeitig Kritik und Weiterentwicklung der \enquote{Information
Literacy} darstellt, wird unter dem Begriff \enquote{Critical Library
Instruction} zusammengefasst. Es geht darum, die positivistischen
Einschränkungen der herkömmlichen Informationskompetenz-Angebote in
Bibliotheken zu überwinden. Information und Systeme, die Informationen
produzieren (insbesondere die Wissenschaft), werden als politisch
verstanden. Es wird, im Rückgriff auf sehr verschiedene kritische
Theorien, versucht, eine pädagogische Praxis zu etablieren, welche die
Lernenden dabei unterstützt, diese politische Sphäre mitzureflektieren.
Im deutschsprachigen Raum praktisch ignoriert, ist die Literatur zur
Critical Library Instruction in den letzten Jahren gewachsen. Eamon C.
Tewell untersuchte die Praxis von Bibliothekarinnen und Bibliothekaren,
welche sich selber als kritisch verstehen (Tewell, Eamon C.: The
Practice and Promise of Critical Information Literacy. Academic
Librarians' Involvement in Critical Library Instruction. In:
\emph{College \& Research Libraries} 79 (2018) 1,
\url{https://doi.org/10.5860/crl.79.1.10}). Dieser Text ist gut als
Übersicht zu nutzen, wenn auch etwas langatmig. In ihrer Sammelrezension
besprechen Lua Gregory und Shana Higgins den Grossteil der relevanten
aktuellen Monographien zum Thema (Gregory, Lua ; Higgins, Shana:
Critical Information Literacy in Practice: A Bibliographic Review Essay
of Critical Information Literacy, Critical Library Pedagogy Handbook,
and Critical Literacy for Information Professionals. In:
\emph{Communications in Information Literacy} 11 (2017) 2, Article 10,
\url{https://doi.org/10.15760/comminfolit.2017.11.2.10}) Über die
Grenzen dieses Ansatzes denkt Jonathan T. Cope anhand eines Vorfalls in
seiner Bibliothek nach. In einer Rechercheeinführung für Studierende
setzte ein Student durch, dass nicht zu einem akademischen Thema,
sondern zu einer rechtsextremen Verschwörungstheorie recherchiert würde
-- offensichtlich nicht mit dem Ziel, über sie zu lernen, sondern andere
von ihr zu überzeugen. Cope, der -- einem Grundsatz der Critical
Information Literacy folgend -- sonst versucht, die Interessen der
Studierenden aufzunehmen, um Lernmöglichkeiten zu schaffen, war von
diesem Vorgehen überwältigt. Er analysiert, dass sich Critical
Information Literacy zu sehr auf liberale Theorien des öffentlichen
Diskurses (Habermas, Rawls et cetera) stützen würde, wenn eine
politischere Theorie, die ideologische Konflikte auch als solche
benennbar, und damit nutzbar, machen würde, nötig sei. Ansonsten würde
auch Critical Information Literacy nur in einem falschen
Neutralitätsverständnis verharren, dass gesellschaftliche Strukturen
reproduzieren und nicht verändern würde. (Cope, Jonathan T.: The
Reconquista Student: Critical Information Literacy, Civics, and
Confronting Student Intolerance. In: \emph{Communications in Information
Literacy} 11 (2017) 2, Article 2,
\url{https://doi.org/10.15760/comminfolit.2017.11.2.2}) (ks)

\hypertarget{monographien}{%
\section{Monographien}\label{monographien}}

Die Machtbeziehungen in Reference Interviews und anderen Kontakten
zwischen Bibliothekar*innen und Nutzer*innen (vor allem Studierenden) in
Beratungen, alltäglichen Kontakten und Lehrveranstaltungen sowie die
Ziele dieser Kontakte sind Thema einer Monographie. (Accardi, Maria T.
(Hrsg.): \emph{The Feminist Reference Desk: Concepts, Critiques, and
Conversations} (Series on Gender and Sexuality in Information Studies,
8). Sacramento: Library Juice Press, 2017) Unter dem Schlagwort
Feminismus (und feministische Kritik) wird hier praktisch das Ziel einer
anderen Welt (also eines anderen Ziels bibliothekarischer Arbeit)
verstanden, in der die Nutzer*innen empowert werden, ihre Potentiale zu
entfalten. Interessant ist das Buch vor allem durch die aufgezeigten
Möglichkeiten, diese den Bibliotheksalltag prägenden Kontakte zu
analysieren, zu kritisieren und auch neu zu fassen. (ks)

Bibliotheken beschäftigen sich seit einigen Jahren damit, vor allem
Entwicklungsprojekte (neue Strategien, Entwurf von Angeboten et cetera)
partizipativ zu gestalten. Gleichzeitig (siehe Schuldt, Karsten ;
Mumenthaler, Rudolf (2017): Partizipation in Bibliotheken. Ein
Experiment, eine Collage. In: \emph{LIBREAS. Library Ideas} 13 (2017),
\url{http://libreas.eu/ausgabe32/schuldt/}) hinterlassen diese Versuche
auch einige offene Fragen. Eine Dissertation von Anja Piontek (verlegt
als eigenständige Monographie) zeigt, dass praktisch die gleichen
Fragen, die gleichen Erwartungen und Kritiken im Museumsbereich
existieren, allerdings schon weiter fortgeschritten. (Piontek, Anja:
\emph{Museum und Partizipation. Theorie und Praxis kooperativer
Ausstellungsprojekte und Beteiligungsangebote} (Edition Museum).
Bielefeld: transcript Verlag, 2017) Interessanterweise verbindet sie
ihre Ergebnisse mit der Argumentation, dass Museen irrelevant würden,
wenn sie nicht partizipativ arbeiten. Das Buch liefert eine breite
Diskussion der Möglichkeiten und Probleme partizipativer Projekte in
Museen, inklusive der Diskussion verschiedener theoretischer Modelle zu
deren Kategorisierung und dreier Case Studies. Es liefert auch für
Diskussionen im Bibliotheksbereich eine gute Grundlage, da immer wieder
theoretische und praxisorientierte Überlegungen zusammengeführt werden.
(ks)

Im Dezember 2017 wurde eine Studie über die Entwicklung des
Open-Access-Publizierens in Großbritannien veröffentlicht: „Monitoring
the transition to open access: December 2017``. Die Untersuchung setzt
Schwerpunkte in den Bereichen (Förder-)Angebote für AutorInnen zum
OA-Publizieren, Anteil von OA-Publikationen am
Gesamtpublikationsaufkommen und Nutzungsstatistiken für OA-Inhalte.
Zudem wird diskutiert, welche Implikationen diese Entwicklungen für
Forschungsförderer und Universitäten zum einen und für
Fachgesellschaften zum anderen haben. Leider sind die im Bericht
verlinkten Anhänge (zur Methodik beziehungsweise zu Kostenabschätzung)
bisher nicht veröffentlicht (Stand: 26.03.2018). (Bericht: Jubb, Michael
et al.: \emph{Monitoring the transition to open access: December 2017},
London: Universities UK, ISBN 978-1-84036-390-6,
\url{http://www.universitiesuk.ac.uk/policy-and-analysis/reports/Documents/2017/monitoring-transition-open-access-2017.pdf})
(mv)

Ähnlich der Open-Access-Idee für wissenschaftliche Publikationen wird
mit wachsendem Engagement ein offener Zugang auch zu Forschungsdaten
diskutiert. Im Kapitel \enquote{Offener Zugang zu Forschungsdaten}
seines Buches \emph{Eine Reputationsökonomie: Der Wert der Daten in der
akademischen Forschung} führt Benedikt Fecher unter anderem einige
wissenschaftsethische Argumente für eine offene Verfügbarmachung von
Forschungsdaten an. Er beruft sich dabei auf Robert K. Mertons Aufsatz
\emph{The Normative Structure of Science} und die von diesem darin
bestimmten vier Eigenschaften akademischer Forschung:
\enquote{organisierter Skeptizismus, Uneigennützigkeit, Universalismus,
Kommunitarismus}. Für die freie Zugänglichmachung von Forschungsdaten
sprechen danach erstens das Prinzip des Skeptizismus im Sinne eines
Anspruchs auf Reproduzierbarkeit und Prüfbarkeit von Forschung, die
notwendigerweise auch eine umfängliche Forschungsdatentransparenz
erfordert. Sowie zweitens das Prinzip des Kommunitarismus, das besagt,
dass sämtliche wissenschaftlichen Ergebnisse der wissenschaftlichen
Gemeinschaft verfügbar sein sollten. Also auch die Forschungsdaten. Und
drittens lässt sich noch das Prinzip der Uneigennützigkeit anführen, das
empirisch durchaus feststellbaren Eigentumsansprüchen von Forschenden in
Bezug auf ihre Daten entgegengehalten werden könnte. (Benedikt Fecher:
Offener Zugang zu Forschungsdaten. In: derselbe: \emph{Eine
Reputationsökonomie: Der Wert der Daten in der akademischen Forschung.}
Wiesbaden: Springer VS, 2018. S. 23--35) (bk)

Auf der Basis der Tagung des Wolfenbüttler Arbeitskreises für
Bibliotheks-, Buch- und Mediengeschichte 2015 werden in
\enquote{Volksbibliothekare im Nationalsozialismus} neun biographische
Schilderungen vorgenommen, ergänzt durch einen einführenden
Literaturbericht, eine Darstellung zu Evangelischen Büchereien in
Württemberg während des Nationalsozialismus und zum dänischen
Bibliothekssystem unter deutscher Besatzung. Die Biographien zeigen, wie
schon für Wissenschaftliche Bibliotheken und auch für andere Bereiche
der deutschen (und österreichischen) Gesellschaft gezeigt wurde, eine
grosse, zum Teil überzeugte, zum Teil zurückhaltende
Kollaborationsbereitschaft; Versuche, die Situation für eigene Karrieren
zu nutzen und wenn, dann eher hinhaltende Handlungen (zum Beispiel das
\enquote{Wegschliessen von Büchern} anstatt der Makulatur) statt
Widerstand. Gefördert wurde dies durch die allgemeinen Machtkämpfe und
Kompetenzschwierigkeiten innerhalb des nationalsozialistischen Staates.
Die kirchlichen Bibliotheken zeigten dabei noch das grösste
Eigeninteresse, welches zu einigen Freiräumen führte. Zudem zeigt sich,
dass in vielen, aber nicht allen, Fällen eine Fortsetzung der Karrieren
von nationalsozialistisch orientierten Volksbibliothekaren in den
Folgegesellschaften möglich war. Das Buch berichtet auch von einer Idee
der Bibliothek als Erziehungsinstitution, welche heute aufgegeben ist.
Interessant wäre, den Blick auch auf die von der Volksbibliothek
\enquote{bekämpften} Leihbibliotheken zu richten. Zudem wird in den
Beiträgen für das vollständige Verständnis ein bestimmtes Vorwissen zur
Geschichte des Volksbüchereiwesens in Deutschland (Stichwort:
Richtungsstreit) vorausgesetzt: So wird zum Beispiel mehrfach berichtet,
dass die beiden wichtigen volksbibliothekarischen Zeitschriften
eingestellt und eine eigene (\emph{Die Bücherei}) gegründet wurde, aber
nur von einer der beiden überhaupt der Name genannt. (Ein Beitrag ist
unverständlicherweise ohne jeden Absatz gesetzt.) (Kuttner, Sven ;
Vodosek, Peter (Hrsg.) (2017): \emph{Volksbibliothekare im
Nationalsozialismus: Handlungsspielräume, Kontinuitäten,
Deutungsmuster}. (Wolfenbütteler Schriften zur Geschichte des
Buchwesens, 50) Wiesbaden: Harrassowitz Verlag, 2017) (ks)

\hypertarget{social-media}{%
\section{Social Media}\label{social-media}}

Elsevier spendet 1 Mio. Pfund an die Universität Oxford
(\url{https://twitter.com/oxford_thinking/status/953205203385364480}).
@StephenEglen weist darauf hin, dass die Universität kürzlich den
Lizenzvertrag mit Elsevier verlängert hatte
(\url{https://twitter.com/StephenEglen/status/953335229850247170}).
Sicher nur ein Zufall... (mv)

@RickyPo weist auf Details zum neuen \enquote{Accelerated Publication
Programme} von Taylor \& Francis hin
(\url{https://twitter.com/RickyPo/status/953990396996149248}): Der
Verlag entlohnt Reviewer für ein schnelles Review mit \$150 --
unabhängig vom Ausgang des Reviews (accept/reject). AutorInnen können
zwischen \enquote{Fast Track} und \enquote{Rapid Track} wählen, der
Preis für diesen \enquote{Prioritized Service} bemisst sich jeweils am
Seitenumfang. Details zu dem Programm sind auf der Seite des Verlages
nachzulesen:
\url{http://taylorandfrancis.com/partnership/commercial/prioritized-publication/}.
(mv)

Wikimedia und Bibliotheken/Archive/Museen -- finden Topf und Deckel
zueinander? Wikimedia Commons ruft GLAM-Institutionen auf, sich dabei
einzubringen, das Handling von Metadaten auf Wikimedia-Plattformen zu
verbessern. via @subsublibrary
(\url{https://twitter.com/subsublibrary/status/959071409690562560}) (mv)

@ernestopriego bringt es auf den Punkt:
\enquote{Bibliometrics/scientometrics are never neutral. Nothing ever
is. It's not said enough.}
(\url{https://twitter.com/ernestopriego/status/960765717334429697}).
(mv)

\hypertarget{konferenzen-konferenzberichte}{%
\section{Konferenzen,
Konferenzberichte}\label{konferenzen-konferenzberichte}}

Zur DHd 2018, der Jahrestagung der Digitale Humanities im
deutschsprachigen Bereich, die vom 26.02.--02.03.2018 in Köln stattfand
und sich ganz kantianisch an einer \enquote{Kritik der digitalen
Vernunft} versuchte, existieren zahlreiche Tagungsberichte und Social
Media Auswertungen (siehe Jörg Wettläufer: Bericht DHd 2018 Köln „Kritik
der digitalen Vernunft\enquote{, 26.2.-2.3. \#dhd2018
\url{http://digihum.de/blog/2018/03/11/bericht-dhd-2018-koeln-kritik-der-digitalen-vernunft-26-2-2-3-dhd2018/},
Twitter Visualisierung von @esthet1cs:
\url{https://twitter.com/esthet1cs/status/970101082277011456} ) Die
Keynote der Philosophieprofessorin Sybille Krämer (FU Berlin) zum
„Stachel des Digitalen -- ein Anreiz zur Selbstreflektion in den
Geisteswissenschaften?} wurde unter anderem von Sophie Schneider in
ihrem Infolog lesenswert besprochen. (Schneider, Sophie (2018): DHd 2018
-- Keynote \#1.
\url{https://informationswissenschaftliches.blog/2018/03/26/dhd-2018-keynote-1/})

Ansonsten fiel die Konferenz vor allem durch ein neues Format auf, den
\enquote{Fight Club}. Statt eines traditionellen Podiums traten in einem
Kölner Club, wo sonst eher Technobeats zu hören sind, vier
Wissenschaftler/innen gegeneinander an. Jeweils vier Fragen wurden an
Hubertus Kohle, Mareike König, Henning Lobin und Heike Zinsmeister
gerichtet, mit je drei Minuten Zeit zum Antworten. Mehr über dieses
Format, aber nicht wer nun gewonnen hat, ist unter anderem bei Jürgen
Hermes nachzulesen. (Hermes, Jürgen (2018): Fragmente \#dhd2018
\url{https://texperimentales.hypotheses.org/2462\#FightClub})

Einen Bericht auf Englisch mit dem schönen Titel \enquote{If Kant used a
Computer\ldots{}}, findet man auf dem Infoditex-Blog. (Arnold, Matthias
; Nunn, Christopher (2018): If Kant used a computer\ldots{} \#DHd2018
(\enquote{Digital Humanities im Deutschsprachigen Raum}),
\url{https://www.infoditex.de/if-kant-used-a-computer/})

Und eine Auswertung der Auswertung liefert Fabian Cremer in seinem sehr
lesenswerten Blogbeitrag: \enquote{Nun sag, wie hältst Du es mit dem
Digitalen Publizieren, Digital Humanities? Eine retrospektive
Gretchenfrage an die \#DHd2018}, der die Dokumentations- und
Veröffentlichungspraxis der Konferenz und ihrer Teilnehmer/innen
kritisch hinterfragt (das Book of Abstracts als Druckwerk für 5€ oder
als PDF ohne Strukturdaten, es wurde zu wenig gebloggt et cetera).
(Cremer, Fabian (2018): Nun sag, wie hältst Du es mit dem Digitalen
Publizieren, Digital Humanities?,
\url{https://editorial.hypotheses.org/113}) (lf)

\hypertarget{populuxe4re-medien-zeitungen-radio-tv-etc.}{%
\section{Populäre Medien (Zeitungen, Radio, TV
etc.)}\label{populuxe4re-medien-zeitungen-radio-tv-etc.}}

Am 11.12.2017 berichtete Margaret Coker in der New York Times über die
Lage in Mossul nach Abzug des IS und beschreibt unter anderem, dass
Kämpfer des IS vor dem Abzug aus der Stadt die Universitätsbibliothek
inklusive der Manuskriptsammlung vernichteten. (Margaret Coker: After
Fall of ISIS, Iraq's Second-Largest City Picks Up the Pieces. In:
\emph{New York Times}, December 11, 2017, S. A6.
\url{https://www.nytimes.com/2017/12/10/world/middleeast/iraq-isis-mosul.html})
(bk)

Die seit einigen Jahren in den USA zu beobachtende \enquote{Opioid
Crisis} wirkt sich auch auf die Arbeit öffentlicher Bibliotheken aus,
wie Annie Correal in der New York Times in ihrer Ausgabe vom 01.03.2018
berichtet. Neben einem verstärkten Einsatz von Sicherheitskräften sollen
Bibliotheken in besonders betroffenen Gebieten mit
Naloxon-Ersthilfe-Sets und Schulungen versorgt werden, damit die
Mitarbeiter*innen im Notfall kompetent lebensrettende Maßnahmen
durchführen können. (Annie Correal: Stocking an Antidote at the
Reference Desk. In: \emph{New York Times}, March 1, 2018, S.A19. Online:
\url{https://www.nytimes.com/2018/02/28/nyregion/librarians-opioid-heroin-overdoses.html})
(bk)

Eine kurze Kolumne im Feuilleton der Frankfurter Allgemeinen Zeitung vom
30.09.1993 zeigte sich sehr von einem neu aufkommenden \enquote{Berliner
Lesehunger} überrascht. Von der Berliner Senatsverwaltung war offenbar
zu erfahren, dass in den Ostberliner Neubauvierteln von Hellersdorf,
Hohenschönhausen und Marzahn eine Zunahme der Entleihungen aus
öffentlichen Bibliotheken um \enquote{50 bis 80 Prozent} verzeichnet
werden konnte. Zugleich sprach sich der damalige Berliner Kultursenator
Ulrich Roloff-Momin für weitere Schließungen aus, da Gesamtberlin mit
264 Bibliotheken für 3,5 Millionen Einwohner überversorgt sei.
(Rietzschel, Thomas: Berliner Lesehunger, in FAZ 30.09.1993, 35) (bk)

Die Potsdamer Stadt- und Landesbibliothek sah sich im Januar 2018
kurzzeitig einer Debatte darüber ausgesetzt, ob sie Bücher mit so
genannten neurechtem Inhalten überhaupt beziehungsweise wie bisher
gehandhabt unkommentiert anbieten soll. Marion Mattekat, Leiterin der
Einrichtung, erklärte unter anderem, dass solche Bücher dann automatisch
ins Programm kämen, wenn sie Bestseller seien. Barbara Lison vom
Deutsche Bibliotheksverband verteidigte die Bereitstellung solcher Titel
durch öffentliche Bibliotheken mit dem Argument der freien
Meinungsbildung. \enquote{Es gehöre zum Selbstverständnis von
Bibliotheken, die pluralistische Entwicklung der Gesellschaft
abzubilden.} (Kramer, Henri: Bibliothek verteidigt Angebot kontroverser
Bücher. In: \emph{Potsdamer Neueste Nachrichten}, 17.01.2018, S. 8.
online: \url{http://www.pnn.de/potsdam/1250176/}) {[}Zu den Grenzen
dieses Ansatzes siehe auch den oben zum Thema Critical Library
Instruction referierten Beitrag von Jonathan T. Cope.) (bk)

\hypertarget{abschlussarbeiten}{%
\section{Abschlussarbeiten}\label{abschlussarbeiten}}

{[}Diesmal keine Hinweise.{]}

\hypertarget{weitere-medien}{%
\section{Weitere Medien}\label{weitere-medien}}

Microsoft Academic macht Google Scholar und Web of Science
beziehungsweise Scopus gleichermaßen ernstzunehmende Konkurrenz -- das
zumindest meinen Sven E. Hug und Martin P. Brändle: Im LSE Impact Blog
stellen sie die wissenschaftliche Suchmaschine vor und erläutern, warum
sie insbesondere für bibliometrische Analysen interessant sein kann. Der
Blogbeitrag als solches ist eine Zusammenfassung eines
Scientometrics-Artikels von 2017 über den Einsatz von Microsoft Academic
für Zitationsanalysen. (Blogbeitrag: Sven E. Hug, Martin P. Brändle:
Microsoft Academic is on the verge of becoming a bibliometric
superpower. LSE Impact Blog, 19.07.2017,
\url{http://blogs.lse.ac.uk/impactofsocialsciences/2017/06/19/microsoft-academic-is-on-the-verge-of-becoming-a-bibliometric-superpower/};
Artikel in Scientometrics:Hug, S. E., Ochsner, M., \& Brändle, M. P.
(2017). Citation analysis with microsoft academic. Scientometrics,
111(1), 371--378. \url{https://doi.org/10.1007/s11192-017-2247-8}
{[}Paywall{]} {[}OA-Version: \url{https://arxiv.org/abs/1609.05354}{]})
(mv)

OpenRefine ist ein nützliches Werkzeug für die Exploration, Korrektur
beziehungsweise Aufbereitung und Anreicherung von Daten (\enquote{A
free, open source, power tool for working with messy data.}, vgl.
\url{http://openrefine.org/}). Es wird zunehmend auch im
bibliothekarischen Umfeld eingesetzt und dankenswerterweise wächst die
Anzahl von freien How-To-Dos. So gibt es etwa bei Histhub eine
Artikelreihe zu OpenRefine
(\url{https://histhub.ch/cat/net/blog/openrefine/}). Eine weitere
nützliche Sammlung von Anleitungen für den bibliothekarischen Alltag
bietet \enquote{Library Carpentry OpenRefine}
(\url{https://data-lessons.github.io/library-openrefine/}), unter
\url{https://github.com/OpenRefine/OpenRefine/wiki/External-Resources}
werden Tutorials verschiedener Art gelistet. (mv)

In seiner jetzt zehnten Ausgabe erschien der Bericht
\enquote{Perceptions 2017: An International Survey of Library
Automation} zur Zufriedenheit von Bibliotheken mit der von ihnen
genutzten Software und den Anbietern dieser Software. Wie immer ist es
auch -- für den englischsprachigen Raum -- eine Übersicht zur insgesamt
in Bibliotheken vorhandenen Software. (Breeding, Marschall: Perceptions
2017: An International Survey of Library Automation. (Library Technology
Guides) \url{https://librarytechnology.org/perceptions/2017/}) (ks)

Einen Einblick in den Stand des Öffentlichen Bibliothekswesens in
Frankreich liefert, allerdings sehr verklärt und zwischen den Zeichen,
der Bericht \enquote{Voyage au pays des bibliothèques. Lire aujourd'hui,
lire demain}, den Erik Orsenna (Académie française) und Noël Crobin
(Inspecteur général des affaires culturelles) nach einer Art
Inspektionsreise für das Kulturministerium verfassten und der im Februar
2018 publiziert wurde. Er ist typisch für solche Texte von Autoren mit
solchen Titeln aus Frankreich: Bibliotheken werden als Kultur, Kultur
als existenziell für Frankreich an sich begriffen und deshalb sehr
positiv beschrieben (als Orte des Lebens, als Stützen der Demokratie).
Allen Beteiligten wird eine hohe Effizienz und ein grosses Engagement
attestiert. Das alles in einem offiziellen und geschwungenen
Französisch, das viel verdeckt. Zwischen den Zeilen werden aber auch
Probleme (mangelnde Ausstattung und Öffnungszeiten, ineffiziente
Kulturpolitik, starke regionale Unterschiede) sichtbar. Die insgesamt 19
Lösungsvorschläge zielen -- auch schon typisch französisch -- auf
zentrale Steuerung, klare Vorgaben, Zusammenarbeit aller Beteiligten vor
Ort und den Aufbau einer Plattform, auf der alle am Bibliothekswesen
Beteiligten miteinander kommunizieren können. (Orsenna, Erik ; Crobin,
Noël ; Ministère de la Culture (2018) Voyage au pays des bibliothèques.
Lire aujourd'hui, lire demain. {[}Paris: Ministère de la Culture{]},
Février 2018,
\url{http://www.culture.gouv.fr/Espace-documentation/Rapports/Rapport-Voyage-au-pays-des-bibliotheques.-Lire-aujourd-hui-lire-demain}{]}
(ks)

Im Mai 1947 ging der fast 16-jährige spätere Fotograf Dave Heath,
nachdem er im Life Magazine vom 12. Mai 1947 das Foto-Essay Bad Boy's
Story von Ralph Crane (S.107--114) über einen adoptierten Jungen namens
Butch gesehen und in der Geschichte sowie der Ausdrucksform eine
Identifikationsfläche für sein eigenes Leben gefunden hatte, in die
Schulbibliothek der Germantown High School (Philadelphia) um sich das
Buch \emph{Photography is a Language} von John R. Whiting
auszuleihen\emph{.} Mit diesem begann sein ernsthafter Einstieg in das
Medium der Fotografie: \enquote{\enquote{Bad Boy's Story} powerfully
captured Heath's own feeling of abandonment and homelessness while
connecting his budding intuition that photography might have a special
meaning for him. Whiting's book helped explain why and how \enquote{Bad
Boy's Story} worked while underscoring the medium's larger history and
cultural significance}. (Davis, Keith F.: Multitude, Solitude. The
Photographs of Dave Heath. Kansas City, Missouri: Hall Family Foundation
in association with the Nelson-Atkins Museum of Art, New Haven, S.15)
(bk)

%autor

\end{document}
